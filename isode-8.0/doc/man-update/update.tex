\input lcustom

\documentstyle[12pt,sfwmac,tgrind,twoside]{report}

\newcount\volnum
\volnum=0
\input manual

\pagenumbering{roman}

\begin{document}

%%%\readauxfiles

\title {{\huge The ISO\ Development Environment:\\ 
       User's Manual\vskip 1.5em
	\LARGE\sl Update Release}\vskip 1.5em}

\author{Colin~J.~Robbins\\
\and Julian~P.~Onions\\
\and X-Tel Services Ltd\\
}
\date{\vskip 2em
        \ifdraft \versiondate/\\ \tt Draft Version \versiontag/
        \else \today\\
	(Version \isodevrsn/)\fi
}

\maketitle
\newpage

\thispagestyle{empty}
\setcounter{page}{0}
\mbox{}
\newpage

\tableofcontents
\footnotetext[0]{\hskip -\parindent
This document (version \versiontag/)
was \LaTeX set \today\ with \fmtname\ v\fmtversion.}

%%%\listoftables
%%%\listoffigures

\preface
The software described herein has been developed as a research tool and
represents an effort to promote the use of the International 
Organisation for Standardisation (ISO) interpretation of open systems interconnection (OSI),
particularly in the Internet and RARE research communities.

\newpage\section*	{Notice, Disclaimer, and Conditions of Use}\label{license}
The ISODE is openly available but is {\bf NOT\/} in the public domain.
You are allowed and encouraged to take this software and build commercial
products.
However, as a condition of use, you are required to ``hold harmless'' all
contributors.

\noindent
Permission to use, copy, modify, and distribute this software and its
documentation for any purpose and without fee is hereby granted, provided
that this notice and the reference to this notice
appearing in each software module be retained unaltered, 
and that the name of any contributors shall not be used in advertising
or publicity pertaining to distribution of the software without specific
written prior permission.
No contributor makes any
representations about the suitability of this software for any purpose.
It is provided ``as is'' without express or implied warranty.

\vskip 0.15in
\noindent
\begin{small}
{\bf All contributors disclaim all warranties with regard to this
software, including all implied warranties of me\-chan\-ti\-bil\-ity
and fitness. In no event shall any contributor be liable for any
special, indirect or consequential damages or any damages whatsoever
resulting from loss of use, data or profits, whether in action of
contract, ne\-gli\-gence or other tortuous action, arising out of or
in connection with, the use or performance of this software.}
\end{small}

\vskip 0.15in
\noindent
As used above,
``contributor'' includes, but is not limited to:
\begin{quote}
\begin{tabular}{l}
The MITRE Corporation\\
The Northrop Corporation\\
NYSERNet, Inc.\\
Performance Systems International, Inc.\\
University College London\\
The University of Nottingham\\
X-Tel Services Ltd\\
The Wollongong Group, Inc.\\
Marshall T. Rose\\
Colin J. Robbins\\
Julian P. Onions
\end{tabular}
\end{quote}

\newpage\section* {Revision Information}

The ISODE-8.0 software is being released as an upgrade to ISODE-7.0.  
This is the final public release of the ISODE in its current form.

This document is a supplement to the ISODE-7.0 Manual set.


\section*	{Release Information}
This version of ISODE is available from several places:-
\begin{itemize}

\item	Internet\\
If you can FTP to the Internet,
you can use anonymous FTP to the host \verb"uu.psi.com"
\verb"[136.161.128.3]"
to retrieve \compressfile/ in BINARY mode from the \tarplace/ directory.
This file is the \pgm{tar} image after being run through the compress program
and is approximately \compressize/ in size.

\item	NIFTP\\
If you run NIFTP over the public X.25 or over JANET, and are
registered in the NRS at Salford, you can use NIFTP with username
``guest'' and your own name as password, to access \verb"UK.AC.UCL.CS" to
retrieve the file \ukcompressfile/.
This file is the \pgm{tar} image after being run through the compress program
and is approximately \compressize/ in size.

\item	FTAM on the JANET, IXI or PSS\\
The source code is available by FTAM at the University College London
over X.25 using 
\begin{describe}
\item JANET 
(DTE \verb"00000511160013")
\item IXI 
(DTE \verb+20433450420113+)
\item PSS 
(DTE \verb"23421920030013") 
\end{describe}
Use TSEL~\verb"259" (ASCII encoding).
Use the ``anon'' user-identity and retrieve the file 
\compressfile/ from the \uktarplace/ directory.
This file is the \pgm{tar} image after being run through the compress program
and is approximately \compressize/ in size.

The file service is provided by the FTAM implementation in ISODE~6.0 or later
(IS FTAM) and is registered in the pilot OSI Directory below
\begin{quote}\footnotesize\begin{verbatim}
bells, Computer Science, University College London, GB
\end{verbatim}\end{quote}

\end{itemize}

The ISODE version 7.0 manual is also available from these sites:

\begin{itemize}

\item  \Docfile/\\
This is the \LaTeX source for the entire documentation set.
It is a compressed \pgm{tar} image (\DocSize/).

\item  \PSfile/\\
This contains the five volume manual in PostScript format.
It is a compressed \pgm{tar} image (\PSsize/).

\end{itemize}


\newpage\section*	{Discussion Groups}
The Internet open discussion group {\tt ISODE@NISC.SRI.COM\/} is
used as a forum to discuss ISODE.
Contact the Internet mailbox {\tt ISODE-Request@NISC.SRI.COM\/}
to be asked to be added to this list.

\subsection*{Support}
Although the ISODE is not ``supported'' per se, it does have a problem
reporting address, {\tt Bug-ISODE@isode.com\/}.  
Bug reports relating to the release of ISODE are
welcome.   Changes will be incorporated into the ISODE Consortium
releases (see Appendix~\ref{IC}), and not released to the public domain.  

\subsection*{Commercial Support}
X-Tel Services Ltd provide support for the ISODE and associated
packages on an international basis.  X-Tel is a founding member of
the ISODE Consortium and will continue to provide general
assistance and site specific support on a commercial basis.  
X-Tel will also enhance and market ISODE Consortium product. 

\[\begin{tabular}{ll}
Postal address:&	X-Tel Services Ltd.\\
&	University Park,\\
&	Nottingham, NG7 2RD\\
&        UK\\[2em]
Phone&		+44 602 412648\\
Fax&		+44 602 790278\\
EMail&	support@xtel.co.uk\\
\end{tabular}\]

The ISODE CONSORTIUM will maintain a database of companies
offering support of the ISODE Package.


\newpage\section*{Acknowledgements}

Since the start of the project, many many people have contributed to
the ISODE software.  A long list of contributors is given in the ISODE
version 7.0 manual.  In this section, we include thanks to the people
who have contributed to getting {\em this} version of the software ready.

Paul Barker of University College London.
Steve Hardcastle-Kille of ISODE Consortium.
Tim Howes of University of Michigan.
Mark Mattingley-Scott of IBM Deutschland GmbH.
George Michaelson of Univerity of Queensland.
Marshall T. Rose of Dover Beach Consultancy Inc.
Wengyik Yeong of PSI.
Peter Yee of NASA.
Alan Young of Concurrent Computer Corporation. 
The PARADISE project.
X-Tel Services Ltd.


\vspace{0.25in}
{\raggedleft cjr \& jpo \par}
{\raggedright Nottingham, England\\
\ifcase\month
    \number\month\or
    January\or February\or March\or April\or May\or June\or
    July\or August\or September\or October\or November\or December\else
\number\month\fi,
{\oldstyle\number\year}\par}

\newpage
\ifodd\value{page} \else
    \mbox{}
    \newpage
\fi
\pagenumbering{arabic}


\chapter{Introduction}


This document contains a set of updates to the version 7 ISODE manual.
This document refers to the final public version of ISODE, released as
ISODE-8.0.  As the changes to the documentation since the 7.0 release
have been only minor, it would be wasteful to reproduce the entire
manual set.

The following chapters contains a description of the changes to
various parts of the ISODE version 7 manual.

Appendix~\ref{ports} includes a table of hardware platforms and
operating systems this version of ISODE is believed to work on.  This
information is based upon reports given to us.  We do not know how
accurate the information is.

\chapter{The ISODE Tailoring file}

This chapter reflects changes to Chapter 6 of Volume 2 ``Tailoring''.

The sub section titled ``Bridge X.25'' has been removed from
Section~6.1.5 ``Interface Specific Tailoring'', reflecting the removal of the
TP0 bridge from the ISODE.  

In the sub section titled ``General X.25 Tailoring'', the text for the
\verb+x25_dnic_prefix+ parameter has been changed to:

\begin{describe}
\item[\verb"x25\_dnic\_prefix"] If you use either or both of the
preceding two mechanisms 
[\verb+x25_intl_zero+ or \verb+x25_strip_dnic+]
then you must use this variable to
inform ISODE of the local DNIC for your host.

It can contain more than one DNIC, this is only relevant if 
\verb+x25_intl_zero+ has the value \verb"on".  
All DNICs in the list will not
have the leading zero added.  This is useful for private X.25
networks (such as the european IXI) that do not need leading zeros.
Only the first DNIC listed will be used in striping.
\end{describe}

Some new sub sections have been added to Section~6.1.5, these are
included below.

\subsubsection{RTnet-X25/PLUS}

X.25 using the Concurrent RTnet-X25/PLUS needs some tailoring as
described below.  
At least one of option must be configured.
  
\begin{describe}
\item[\verb"x25\_default\_line"] RTnet-X25/PLUS requires attached lines to be
            configured with names.  This variable determines which named
            line will be used if an appropriate match is not found in
            x25\_communities (see below).

\item[\verb"x25\_communities"] pairs of community name and line name to 
	    determine which line will be used for a call to a 
	    particular community. 
\end{describe}


\subsubsection{TP4}
\begin{describe}
\item[\verb"nsap\_default\_stack"]
	If the TP4 interface has access to both CONS and CLNS, this
	parameter is used to determine the default.  The value can be either
	\verb"CONS" or \verb"CLNS".
	The default value is \verb"CLNS".
	You should also consult \man isonsapsnpa(5), which defines how
	the default stack can be overridden for specific NSAP addresses.
\item[\verb"local\_nsap"]
	A default NSAP address for tsapd to listen on if the \verb"-N"
	flag is used! 
\end{describe}

\subsubsection{TLI TP4}
The following setting are only useful when TP\_TLI is defined, that is,
the TP4 is provided by the TLI interface.

\begin{describe}
\item[\verb"tli\_cots\_dev"]
	The name of the device node used to access COTS
	(default \verb"/dev/ositpi").
\item[\verb"tli\_clts\_dev"]
	The name of the device node used to access CLTS
	(default \verb"/dev/ositpi").
\end{describe}

\subsubsection{ICL TLI TP4}
When using TLI on an ICL~DRS6000, the following two addressing 
parameters will need setting:

\begin{describe}
\item[\verb"tli\_initiator\_prefix"]
	The subnet name of the network interface to use for outgoing X.25 calls
	(default \verb"x25_tliin").
\item[\verb"tli\_responder\_prefix"]
	The subnet name of the network interface to use for incomming
	X.25 calls. 
	(default \verb"x25_tlire").
\end{describe}

\subsubsection{XTI TP4}
The following setting is only useful when XTI is defined.  That is TP4
is provided by an XTI interface.  XTI is only alpha test in this release.
\begin{describe}
\item[\verb"xti\_tp0\_ident"]
	The string value is the string passed to the XTI interface during
	t\_bind for X121 addresses.
	Default \verb"TOSITP0".
\item[\verb"xti\_tp4\_ident"]
	The string value is the string passed to the XTI interface during
	t\_bind for TP4 addresses.
	Default \verb"TOSITP4".
\end{describe}


\chapter{QUIPU}

This chapter details changes to the Volume 5 manual: QUIPU.

There have been extensive changes to the Chapter 6: DE.  The new
chapter is included in its entirety as Appendix~\ref{DUA:de}.

There is a new Chapter on ``DM tools''.  These are a set of DISH shell
scripts which provide some simple bulk data management functions using
DAP.

Chapter 10 has a new sub section on the SearchACL attribute syntax, 
this is in Appendix~\ref{searchacl}.

Chapter 11 of Volume 5 has a new section describing searcl ACLs in
more detail.  This is included as Appendix~\ref{disc_sacl} of this document.
  
It should also be noted that the {\tt QUIPU-support@cs.ucl.ac.uk} support
address no longer exists.  This has been replaced with a bug reporting
address.  Any changes required will be incorporated into the ISODE Consortium
releases, and not released to the public domain.  

\begin{quote}
{\tt bug-quipu@isode.com\/}
\end{quote}
However, it should be noted that the open discussion list 
\begin{quote}
{\tt quipu@cs.ucl.ac.uk\/}
\end{quote}
does still exists.  Mail 
\begin{quote}
{\tt quipu-request@cs.ucl.ac.uk\/}
\end{quote}
to join the discussion.

\section{Text Changes}

The second paragraph of Section 17.2.1, page 201 in volume 5 should read:
\begin{quote}
The fields \verb"ca_ext", \verb"ca_progress", \verb"ca_requestor" 
and \verb"ca_aliased_rdns" are provided as
they are defined within X.500.  Neither the QUIPU DSA or DUA use these
fields, but they must be initialized to zero.
\end{quote}
  
The text in Section 17.2.2 ``Results'' should read:
\begin{quote}
The field \verb"cr_aliasdereferenced" is set to \verb"TRUE" if the base
object of the operation was an alias, and was dereferenced.

The field \verb"cr_requestor" is the DN of the requestor of the
operation.  It is only used for secure operations.

The other fields are used by {\em pepsy} whilst encoding and decoding
the structure.
\end{quote}

All of the references to \file{.podrc} in Chapter 9 should be replaced
with \file{.duarc}

\chapter{Known Problems}

This chapter describes some known problems with the current
implementation of ISODE.

\subsubsection{Simply Encoded Data}

If ``user-data'' at the presentation layer is simply encoded, ISODE is
sometimes unable to decode the data.  ISODE always uses fully encoded
data.  A default context in not proposed by any of the applications,
thus simply encoded data is rarely required.

\subsubsection{Large RFC-1006 TPDUs}

If RFC-1006 is used to transfer a large TPDU, then the current
implementation effectively does a read of the RFC-1006 header,
followed by \verb+n+ bytes of data, as specified in the header.

If \verb+n+ is large, it will require several reads to 
actually read this number of bytes. These reads are done
synchronously, one after another until the packet is in. However, if
its a large packet and the TCP connection is unreliable, some of the 64K
may get dropped and have to be retransmitted.
If a router goes down also in this time you can get left hanging.
This particularly effects QUIPU.

\subsubsection{ACSE Initialisation}

During the initialisation stage of an ACSE connection, it is possible
for a connect request to momentarily block.


\appendix

\chapter{DE}
\label{DUA:de}

DE (which stands for {\bf D}irectory {\bf E}nquiries) is a directory user
interface primarily intended to serve as a public access user interface.  It
is a successor to, and borrows something of the style of, the {\em dsc} 
interface released in a previous version.
It is primarily aimed at the novice user, although more sophisticated users
should find that it is flexible enough to answer the majority of queries
they wish to pose.  

DE has more features than those discussed below.  However, the program 
has extensive on-line help as it is envisaged that it will often be used in
environments where neither on-line help nor paper documentation will be
available.

\section {Using DE}

\subsection {Starting up}

DE will work quite happily without any knowledge of the user's
terminal type, assuming a screen size of 80~x~24 in the absence of terminal
type information.  If, however, the user's terminal type is not recognised
by the system, the user will be prompted to try and enter an alternative.
The user can examine a list of valid terminal types; typing \verb+<CR>+ accepts
a terminal type of ``dumb''.

It is possible to configure DE to force confirmation of screen lengths of
greater than 24 lines --- this helps with WAN access as some virtual terminal 
protocols do not propagate the screen size.

\subsection {Searching for a Person}

The interface prompts the user for input with the following four questions:

\begin{quote}\footnotesize\begin{verbatim}
Person's name, q to quit, * to list people, ? for help
:- barker
Dept name, * to list depts, <CR> to search all depts, ? for help
:- cs
Organisation name, <CR> to search `ucl', * to list orgs, ? for help
:- 
Country name, <CR> to search `gb', * to list countries, ? for help
:- 
\end{verbatim}\end{quote}

Note from the above example that it is possible to configure the interface 
so that local values are defaulted: RETURN accepts ``ucl'' for organisation,
and ``gb'' for country.  The above query returns a single result which is
displayed thus:

\begin{quote}\footnotesize\begin{verbatim}
United Kingdom
  University College London
    Computer Science
      Paul Barker
        telephoneNumber       +44 71-380-7366
        electronic mail       P.Barker@cs.ucl.ac.uk
        favouriteDrink        guinness
                              16 year old lagavulin
        roomNumber            G21
\end{verbatim}\end{quote}

If several results are found for a single query, the user is asked to select
one from the entries matched.  For example, searching for ``jones'' in
``physics'' at ``UCL'' in ``GB'' produces the following output:

\begin{quote}\footnotesize\begin{verbatim}
United Kingdom
  University College London

Got the following approximate matches.  Please select one from the 
list by typing the number corresponding to the entry you want.

    1 Faculty of Mathematical and Physical Sciences
    2 Medical Physics and Bio-Engineering
    3 Physics and Astronomy
    4 Psychiatry
    5 Psychology
\end{verbatim}\end{quote}

Selecting ``Physics and Astronomy'' by simply typing the number 3, the
search continues, and the following is displayed:

\begin{quote}\footnotesize\begin{verbatim}
United Kingdom
  University College London
    Physics and Astronomy

Got the following approximate matches.  Please select one from the
list by typing the number corresponding to the entry you want.

     1 C L Jones     +44 71-380-7139
     2 G O Jones     +44 71-387-7050 x3468  geraint.jones@ucl.ac.uk
     3 P S Jones     +44 71-387-7050 x3483
     4 T W Jones     +44 71-380-7150
\end{verbatim}\end{quote}

In this condensed format, telephone and email information is displayed.

\subsection {Searching for other information}

Information for organisations can be found by specifying null entries for 
the person and department.

Information for departments can be found by specifying null input for
the person field.

Information about rooms and roles can be found as well as for people by, for
example, entering ``secretary'' in the person's name field.

\subsection{Interrupting}

If the user wishes to abandon a query or correct the input of a query (maybe
the user has mis-typed a name), {\em control-C} resets the interface 
so that it is
waiting for a fresh query.  Typing ``q'' at prompts other than the person
prompt results in the user being asked to confirm if they wish to quit.
If the user replys ``n'', the interface resets as if {\em control-C} had been
pressed.

\subsection{Quitting}

Type ``q'' (or optionally ``quit'' --- see below) at the prompt for a person's 
name.  Type ``q'' at other prompts, and the user is asked to confirm if they 
wish to quit.  If the use replys ``n'', the interface resets to allow a
query to be entered afresh.

\section {Configuration of DE}

As DE is intended as a public access dua, it is only configurable on a
system-wide basis.
DE installs help files and the \file{detailor} file into a directory 
called \file{de/} under \verb+ISODE+'s ETCDIR.

\subsection{Highly recommended options}

The \file{detailor} file 
contains a number of tailorable variables, of which the
following are highly recommended:

\begin{description}

\item [\verb+dsa\_address+:] This is the address of the access point DSA.
If two or more dsa\_address lines are given, the first dsa\_address is tried
first, the second dsa\_address is tried if connecting to the first address
fails.  Third and subsequent dsa\_address entries are ignored. If there is no
dsa\_address entry in the \file{detailor} file, the first value in the
\file{dsaptailor} file is used.

\begin{quote}\small\begin{verbatim}
dsa_address:Internet=128.16.6.8+17003
dsa_address:Internet=128.16.6.10+17003
\end{verbatim}\end{quote}

\item [\verb+username+:] This is the username with which the DUA binds to 
the Directory.  It is not strictly mandatory, but you are strongly encouraged
to set this up.  It will help you to see who is connecting to the DSA.

\begin{quote}\footnotesize\begin{verbatim}
username:@c=GB@o=X-Tel Services Ltd@cn=Directory Enquiries
\end{verbatim}\end{quote}

\end{description}

\subsection{Variables you will probably want to configure}

You will almost certainly want to set at least some of these to suit your 
local system:

\begin{description}

\item [\verb+welcomeMessage+:]  This is the welcoming banner message.  The 
default is ``Welcome to the Directory Service''.

\begin{quote}\small\begin{verbatim}
welcomeMessage:Welcome to DE
\end{verbatim}\end{quote}

\item [\verb+byebyeMessage+:]  This enables/disables the display of a 
message on
exiting DE.  This variable takes the values ``on'' and ``off''.  The 
message displayed is the contents of the file 
{\em debyebye,} which should be placed in the same directory as all
DE's help files.  The default is not to display an exit message.
 
\begin{quote}\small\begin{verbatim}
byebyeMessage:on
\end{verbatim}\end{quote}
  
\item [\verb+default\_country+:]  This is the name of the country to search by
default: e.g., ``GB''.

\begin{quote}\small\begin{verbatim}
default_country:gb
\end{verbatim}\end{quote}


\item [\verb+default\_org+:]  This is the name of the organisation to search by
default: e.g., ``University College London''

\begin{quote}\small\begin{verbatim}
default_org:University College London
\end{verbatim}\end{quote}

\item [\verb+default\_dept+:] This is the name of the department 
(organisational unit) to search by default: e.g., ``Computing''.  This will 
usually be null for public access duas.

\begin{quote}\small\begin{verbatim}
default_dept:
\end{verbatim}\end{quote}

\end{description}

\subsection{Attribute tailoring}

The following configuration options all concern the display of attributes.
The settings in the \file{detailor} file will probably be OK initially.

\begin{description}

\item [\verb+commonatt+:]  These attributes are displayed whatever type of 
object is being searched for, be it an organisation, a department, or a person.

\begin{quote}\small\begin{verbatim}
commonatt:telephoneNumber
commonatt:facsimileTelephoneNumber
\end{verbatim}\end{quote}

\item [\verb+orgatt+:]  These attributes are displayed (as well as the common
attributes --- see above) if an entry for an organisation is displayed.

\begin{quote}\small\begin{verbatim}
orgatt:telexNumber
\end{verbatim}\end{quote}

\item [\verb+ouatt+:]  These attributes are displayed (as well as the common
attributes --- see above) if an entry for an organisational unit (department)
is displayed.

\begin{quote}\small\begin{verbatim}
ouatt:telexNumber
\end{verbatim}\end{quote}

\item [\verb+prratt+:]  These attributes are displayed (as well as the common
attributes --- see above) if an entry for a person, room or role is displayed.

\begin{quote}\small\begin{verbatim}
prratt:rfc822Mailbox
prratt:roomNumber
\end{verbatim}\end{quote}

\item [\verb+mapattname+:]  This attribute allows for meaningful attribute 
names to be displayed to the user.  The attribute names in the quipu
oidtables may be mapped onto more user-friendly names.  This allows for 
language independence.  

\begin{quote}\small\begin{verbatim}
mapattname:facsimileTelephoneNumber fax
mapattname:rfc822Mailbox electronic mail
\end{verbatim}\end{quote}

\item [\verb+mapphone+:]  This allows for the mapping of international 
format phone
numbers into a local format.  It is thus possible to display local phone
numbers as extension numbers only and phone numbers in the same country
correctly prefixed and without the country code.

\begin{quote}\small\begin{verbatim}
mapphone:+44 71-380-:
mapphone:+44 71-387- 7050 x:
mapphone:+44 :0
\end{verbatim}\end{quote}

\item [\verb+greybook+:]  In the UK, big-endian domains are used in mail 
names.  By setting this variable on, it is possible to display email addresses 
in this order rather than the default little-endian order.

\begin{quote}\small\begin{verbatim}
greyBook:on
\end{verbatim}\end{quote}

\item [\verb+country+:]  This allows for the mapping of the 2 letter ISO 
country codes (such as GB and FR) onto locally meaningful strings such as, for
english speakers, Great Britain and France.

\begin{quote}\small\begin{verbatim}
country:AU Australia
country:AT Austria
country:BE Belgium
\end{verbatim}\end{quote}

\end{description}

\subsection{Miscellaneous tailoring}

There are a number of miscellaneous variables which may be set.

\begin{description}

\item [\verb+maxPersons+:]  If a lot of matches are found, DE will display the
matches in a short form, showing email address and telephone number only.
Otherwise full entry details are displayed.  This variable allows the number
of entries which will be displayed in full to be set --- the default is 3.

\begin{quote}\small\begin{verbatim}
maxPersons:2
\end{verbatim}\end{quote}

\item [\verb+inverseVideo+:]  Prompts are by default shown in inverse video.  
Unset this variable to turn this off.

\begin{quote}\small\begin{verbatim}
inverseVideo:on
\end{verbatim}\end{quote}

\item [\verb+delogfile+:]  Searches are by default are logged to the file
\file{de.log}
in \verb+ISODE+s LOGDIR.  They can be directed elsewhere by using this
variable.

\begin{quote}\small\begin{verbatim}
delogfile:/tmp/delogfile
\end{verbatim}\end{quote}

\item [\verb+logLevel+:]  The logging can be turned off. It can also be turned
up to give details of
which search filters are being successful --- this will hopefully allow some
tuning of the interface.

\begin{itemize}
\item Level 0 --- turns the logging off.
\item Level 1 (the default level) --- logs binds, searches, unbinds
\item Level 2 --- gives level 1 logs, and logging analysis of which
filters have been successful and which failed
\end{itemize}

\begin{quote}\small\begin{verbatim}
logLevel:2
\end{verbatim}\end{quote}

\item [\verb+remoteAlarmTime+:]  A remote search is one where
the country and organisation name searched for not the same as
the defaults.  If the search has not completed within a configurable number
of seconds, a message is displayed warning the user that all may not be well.
The default setting is 30 seconds.
The search, however, continues until it returns or is interrupted by the
user.

\begin{quote}\small\begin{verbatim}
remoteAlarmTime:30
\end{verbatim}\end{quote}

\item [\verb+localAlarmTime+:]  As for {\em remoteAlarmTime}, except for 
local searches. The default setting is 15 seconds.

\begin{quote}\small\begin{verbatim}
localAlarmTime:15
\end{verbatim}\end{quote}

\item [\verb+quitChars+:]  The number of characters of the word ``quit'' 
which a user must type to exit.  The default setting is 1 character.

\begin{quote}\small\begin{verbatim}
quitChars:1
\end{verbatim}\end{quote}

\item [\verb+allowControlCtoQuit+:]  This enables or disables the feature where
a user may exit the program by typing \verb+control-C+ at the prompt for a 
person's name.  The default setting is on.

\begin{quote}\small\begin{verbatim}
allowControlCtoQuit:on
\end{verbatim}\end{quote}

\item [\verb+wanAccess+:] This enables the feature where a user is asked to
confirm that the size of their terminal is really greater than 24 lines.
This helps with telnet access if the screen size is not propagated.  The
default setting is off.

\begin{quote}\small\begin{verbatim}
wanAccess:on
\end{verbatim}\end{quote}

\end{description}

\section{Dynamic tailoring}

It is possible for a user to modify some variables used by DE while
running the program.  In particular, this allows a user to recover from a
situation where the terminal emulation is not working correctly --- an
apparently frequent occurrence!

Dynamic tailoring of variables is offered by use of the SETTINGS help screen.
Typing {\tt ?settings} at any prompt will display the current settings of
dynamically alterable variables.  The user is then offered the opportunity
of modifying the variables.  Variables which may currently be altered in
this way are:

\begin{description}

\item [\verb+termtype+] The user's terminal type, as set in the UNIX ``TERM''
environment variable.

\item [\verb+invvideo+] Turn inverse video ``on'' (if the terminal 
supports it) or ``off''

\item [\verb+cols+] Set the width of the screen to a number of columns

\item [\verb+lines+] Set the length of the screen to a number of lines

\end{description}


\chapter{DM Tools}
\label{DUA:dmtools}

\pgm{DM tools} are a set of shell scripts which provide some simple bulk
data management functions using DAP.  The tools have the following
characteristics.

\begin {itemize}
\item They obviate such skulduggery as
editing EDB files on the DSA machine.
\item They provide some ability to add, modify and delete entries and
attributes.
\item They will play a part in managing data from multiple sources, but
there are several limitations (see caveats later).
\item They will not handle large numbers of thousands of entries in one go,
but have been used with success with a few thousand entries.
\item Based on DISH commands with lashings of shell and [gn]awk.
\end{itemize}

\section {How the Tools Work}

The tools are driven by data in a syntax very similar to the EDB files.  A
special-purpose difference tool is used to work out differences between the
current version of the data and the previous version.  Another tool
processes the resultant differences (which may, of course, be the original
file the first time round) and translates this data into a shell script of
the DISH commands required to update the directory appropriately.  Run the
resultant shell script to apply the modifications.

\section {The Bulk Data Format --- dmformat}

This is very similar to the EDB format.  The differences are as follows:\\

%%%\renewcommand{\arraystretch}{2}
\begin {tabular}{|l|l|}
\hline
EDB & DMFORMAT \\
\hline
\hline
DIT hierarchy mapped & Flat file with embedded info \\
onto UNIX directory & saying where entries should be  \\
structure & loaded in the DIT \\
& \\
Files start with: & File don't start with ... \\
MASTER & \\
date in UTC format & \\
&                                    File contains "rootedAt" info \\
& \\
&                                    Syntax includes mechanism for \\
&                                    specifying deletion of an entry / \\
&                                    attribute \\
& \\
Can only represent one set&          Can represent information \\
of sibling entries &                 in an entire subtree or \\
&                                    collection of subtrees \\
\hline
\end {tabular}\\

Comments may be interspersed throughout the file.  A comment line begins
with a ``\#'' character.

rootedAt indicates the parent node in the DIT for subsequent entries in the
file.  Separate a rootedAt line from entries by one or more blank lines.

A set of entries follows a rootedAt line.  These are formatted in the same
way as in an EDB file: i.e., an entry is a sequence of attribute type-value
pairs, where the first pair is the RDN for the entry.

Entries are separated from other entries by blank lines.

In addition to the conventional syntax it is possible to specify deletion of
entries and attributes.
\begin{itemize}
\item Specify entry deletion by prefixing the RDN with the ``!'' character.
\item Specify attribute value deletion by prefixing the attribute type=value
line with a ``!'' character.
\end{itemize}

A file can contain information for many DIT subtrees by including more
rootedAt lines.

\section{dmformat --- An Example}

\begin{verbatim}
#subsequent entries are relative to this point
# in the DIT
rootedAt= c=gb@o=UCL@ou=CS

# add this entry with these attributes
#   if it doesn't already exist
# try to add in these attribute values if
#   the entry already exists
cn=Paul Barker
surname=Barker
telephoneNumber=+44 71 380 7366
objectClass=organizationalPerson & quipuObject & ...

# Add the first telephone number attribute
# value and delete the second
cn=Steve Kille
telephoneNumber=+44 71 380 7294
!telephoneNumber=+44 71 380 1234

# Delete this entry
!cn=Colin Robbins
# don't have to supply attributes, but can
# if you like
!telephoneNumber=+44 71 387 7050 x3688

#subsequent entries are relative to this point
# in the DIT
rootedAt= c=gb@o=UCL@ou=Physics

\end{verbatim}

\section{Using the Tools}

The tools can be used to load the database initially as follows:

\begin{itemize}

\item Produce a file ``newfile'' of entries to be loaded
\item Make a file of DISH operations to effect the update

\verb|crmods < newfile|

\item Apply the updates

\verb|sh modfile|
\end{itemize}

It can also be used for subsequent amendments

\begin{itemize}
\item Create a file of difference data

\verb|dmdiff oldfile newfile > difffile|

\item Create a shell/DISH script to do the update

\verb|crmods < difffile|

\item Apply the updates

\verb|sh modfile|
\end{itemize}

There are examples of using the tools and sample Makefiles in the README
file accompanying the software.

\section{Preparing Data for use with DM Tools}

The tools will work more efficiently if the following guidelines are
followed:

\begin{itemize}
\item Attribute type strings in DM files should be the same as those
written out by DISH when using ``showentry -edb''

In practice this means using the abbreviated attribute names as specified in
\$(ETCDIR)/oidtable.at.  E.g., use ``cn'' rather than ``commonName'', and
``mail'' rather than ``rfc822Mailbox''.

\item Be consistent with capitalisation and case in general between DM files
produced from the various sources.

\item Attribute values with DN syntax should have the country name part
represented in capitals, as in ``c=GB''.  This is because QUIPU always
writes them out that way.  In all other cases, QUIPU maintains the case
with which entries' attributes are created.

\end{itemize}

\section{Some Specific Shortcomings of the DM Tools}

\begin{itemize}
\item Scale --- the shell script, {\tt modfile,} which crmods produces, is
very large for substantial amounts of data or data differences

It may be more manageable to split data into a set of department files, as
for EDBs, and apply set of updates.

\item Matching of attribute types and attribute values is case-sensitive,
whereas almost always it should be case-independent.

In practice this is not too much of a problem

\begin{itemize}
\item At worst, it means that too many ``differences'' are discovered
\item QUIPU does the ``right thing'' anyway
\end{itemize}

\item No explicit mechanism for renaming entries --- achieved by deleting
entry with old name and creating a new entry.

You may thus discard attribute information which has been loaded from
another source.

\item Tools have no knowledge that entries may be mastered by more
than one source.

If an entry is deleted from one source, it will be deleted from the
Directory even if the entry still exists in another source.  This may, or
may not, be want you want!

\item No explicit support for maintenance of seeAlso, roleOccupant and other
attributes which have DN syntax.

All necessary management to avoid ``dangling pointers'' must be
achieved externally

\item No support for management of aliases

\item Updating over DAP can be rather slow for entries with large numbers of
siblings (in QUIPU terms, in a large EDB file).

There is a solution --- use the TURBO\_DISK option when compiling
QUIPU.  This makes use of GNU's gdbm package.  Consider this if you do
a lot of updating and you have large EDB files.

\item There are some known bugs.  Inherited attributes are not always handled 
correctly, and problems with eDBInfo have been reported.
%%% Fixes gratefully received --- send them to \verb+<p.barker@cs.ucl.ac.uk>.+

\end{itemize}

\section {General Data Management Problems Not Catered For}

\begin{itemize}
\item Management of data from multiple sources is very difficult --- no support
for merging data from different sources, or for consistent deletion.
\item No framework for discrimination between quality of data sources --- this
must be handled manually
\item Relying on diffs not really satisfactory --- need to rebuild database
periodically from source data
\item Naming of entries --- DM tools offer no help with naming to person
maintaining the Directory
database.  This administrator should be aware of at least the following
problems
\begin{itemize}
\item Two sources may name an entity differently

\begin{verbatim}
source one: P Barker
source two: Paul Barker
\end{verbatim}

\item Need to be careful that no duplicate RDNs are formed when processing
the source data into EDBs or DM files.
\begin{itemize}
\item If building EDBs, QUIPU will detect multiple RDNs as it loads its
database.
\item DM tools will perform multiple updates on a single entry
\end{itemize}

\item Even in case where one is loading from a single source, the name which
is systematically derivable may be unsatisfactory. E.g.,

\verb|PHYS & ASTRO|

rather than

\verb|Physics and Astronomy|

\item A source's vies of what constitutes a department may be parochial,
suiting particular requirements.  For example, the UCL telephone directory
database has the following two departments

\begin{verbatim}
BIOLOGY (DARWIN)
BIOLOGY (MEDAWAR)
\end{verbatim}

whereas the University view, which must be represented, is that there is
just a single ``Biology'' department

\item Need to be careful when joining departments in this way that no RDN
clashes occur.  If they do occur, a solution is to name entries with
multiple value RDN.

cn=Fred Bloggs\%ou=Biology (Medawar)

\end{itemize}

\end{itemize}


\chapter{Attribute Synatxes}

The sections in this appendix should be placed at the end of Section~10.3
``COSINE/Internet Attribute Syntaxes'' of the Version~7.0 manual.

\section{SearchACL}

\label{searchacl}
\index{searchACL attribute}
\begin{center}\small
\begin{tabular}{|l|}\hline
QUIPU Attributes \\ \hline
	searchACL\\
\hline
\end{tabular}
\end{center}
\begin{quote}\begin{verbatim}
<searchaclvalue>::= <aclwho> "#" <access> "#" <attrs> 
                    "#" <scope>
                    [ "#" <max-results> "#" <partialresults>
                    [ "#" <minkeylen> ]]
<access>        ::= "search" | "nosearch"
<attrs>         ::= "default" | <attr-list>
<attr-list>     ::= <attributetype> | <attributetype> 
                    "$" <attr-list>
<scope>         ::= <singlescope> | <singlescope> 
                    "$" <scope>
<singlescope>   ::= "subtree" | "singlelevel" | "baseobject"
<max-results>   ::= <integer>
<partialresults>::= "partialresults" | "nopartialresults"
<minkeylen>     ::= <integer>
\end{verbatim}\end{quote}
The use of the search ACL attribute is discussed in Section~\ref{disc_sacl}.

\section{ListACL}
\label{listacl}
\index{listACL attribute}
\begin{center}\small
\begin{tabular}{|l|}\hline
QUIPU Attributes \\ \hline
	listACL\\
\hline
\end{tabular}
\end{center}
\begin{quote}\begin{verbatim}
<listaclvalue>  ::= <aclwho> "#" <access> "#" <scope>
                        [ "#" <max-results> ]
<access>        ::= "list" | "nolist"
<scope>         ::= "entry" | "children"
<max-results>   ::= <integer>
\end{verbatim}\end{quote}
The use of the list ACL attribute is discussed in Section~\ref{disc_lacl}.

\section{AuthPolicy}
\label{authpolicy}
\index{authPolicy attribute}
\index{authentication policy}
\begin{center}\small
\begin{tabular}{|l|}\hline
QUIPU Attributes \\ \hline
	authPolicy\\
\hline
\end{tabular}
\end{center}
\begin{quote}\begin{verbatim}
<authpvalue>    ::= <modpolicy> "#" <readpolicy> 
                    "#" <searchpolicy>
<modpolicy>     ::= <authpolicy>
<readpolicy>    ::= <authpolicy>
<searchpolicy>  ::= <authpolicy>
<authpolicy>    ::= "trust" | "simple" | "strong"
\end{verbatim}\end{quote}

\chapter{Search Access Control}
\label{disc_sacl}\index{searchACL attribute}
  
The access control described above is sufficient to protect individual
entries from unauthorized access, but it does little to protect the
directory as a whole from ``trawling'': the disclosure of large amounts
of organizational data or structure information by repeated searches.
In the past, the administrative size limit was the only control on such
access.  The search ACL is designed to allow much more flexible control
on the types on searches performed and the number of results that can
be obtained by a directory user.

A search ACL belongs to a single entry and specifies restrictions on
searches involving that entry and possibly its descendants.  A search
ACL scope must be specified.  A scope of ``subtree'' means the search
ACL applies during subtree searches involving the entry and its
descendants.  Note that the subtree search must be rooted at or above
the entry containing the search ACL for the ACL to apply.  A
``singlelevel'' search ACL applies only during a single level search
rooted at the entry containing the ACL.  Note that the subtree and
single level scopes are disjoint:  a subtree search ACL has no bearing
on a single level search and vice versa.

A search ACL with scope ``baseobject'' applies to the entry during any
type of search, and can thus be used to provide discretionary access
control for searches in a way similar to normal access control.

The simplest and most restrictive application of a search ACL is to
prevent searching on certain attribute types.  For example, the following
search ACL would not allow anyone to perform any type of search by the
userPassword attribute in the subtree rooted at the entry containing
the search ACL (or in its children).

\begin{quote}\small\begin{verbatim}
sacl= others # nosearch # userPassword \ 
           # subtree $ singlelevel
\end{verbatim}\end{quote}

The access selector for a search ACL is the same as for a normal QUIPU
ACL.  Note that a search started at a point in the DIT below the entry
containing a search ACL is not constrained by that search ACL.

To allow searches by certain attributes, but to limit the number of
results that can be returned, a search ACL like this may be used:

\begin{quote}\small\begin{verbatim}
sacl= others # search # commonName $ surname \ 
           # subtree # 10 # partialresults
sacl= others # nosearch # default # subtree
\end{verbatim}\end{quote}

This allows others to search only by the attributes commonName and surname,
returning at most 10 matches.  If ``trawling'' is a concern, the search
ACL above can be modified to not return any results if the size limit
specified is exceeded:

\begin{quote}\small\begin{verbatim}
sacl= others # search # commonName $ surname \ 
           # subtree # 10 # nopartialresults
sacl= others # nosearch # default # subtree
\end{verbatim}\end{quote}

Note that both of the preceeding examples only restrict subtree searches.
If single level searches are to be restricted also, the scope should be
changed to ``subtree \$ singlelevel.''  Note also that the attributes not
specified in another search ACL may be referred to by using the ``default''
keyword.  In the example above, this capability is used to disallow
searches on any attributes but commonName and surname.

An individual entry may protect itself from being found by certain types
of searches by using the ``baseobject'' search ACL scope.  For example,

\begin{quote}\small\begin{verbatim}
sacl= others # nosearch # commonName $ surname # baseobject
\end{verbatim}\end{quote}

Finally, it may be desirable to restrict certain types of searches below
an entry.  For example, if not checked, an effective dumping technique is
to do repeated searches of the form cn=a*, cn=b*, etc.  This technique is
not entirely thwarted by the ``nopartialresults'' capability
described above, because a clever and determined attacker can construct
repeated range filters where the range is small enough not to
exceed the size limit.

As a defense against such attacks, a minimum substring key length may
be specified in a searchACL.  This minimum length is also used as the
minumum prefix that must be common to any range queries using the inequality
operators.  For example, a search acl like this one

\begin{quote}\small\begin{verbatim}
sacl= others # search # default # subtree # 10 \ 
           # nopartialresults # 3
\end{verbatim}\end{quote}
  
specifies that others may perform subtree searches by the default
attribute set, returning at most 10 matches.  No matches will be
returned if the limit of 10 is exceeded.  Furthermore, any substring
queries must contain a substring that is at least 3 characters long,
and any inequality range queries must involve values whose first 3
characters are the same.  To see how this works, consider the following
queries and the reason they are either accepted or rejected because
they violate the above search ACL.

\begin{tabular}{lll}
Filter              &       Accepted? & Explanation \\\hline
cn=a*               &       no        & maximum substring length is 1\\
cn=aa*              &       no        & maximum substring length is 2\\
cn=*a*              &       no        & maximum substring length is 1\\
cn=abc*             &       yes       & maximum substring length is 3\\
cn=a*abcd*          &       yes       & maximum substring length is 4\\
(cn$>$=a \& cn$<$=b)     &       no        & common prefix length is 0\\
(cn$>$=aa \& cn$<$=ab)   &       no        & common prefix length is 1 (a)\\
(cn$>$=abcdef \& cn$<$=abcghi) & yes       & common prefix length is 3 (abc)\\
\end{tabular}

\section{List ACL}
\label{disc_lacl}
\index{listACL attribute}

Just as a search ACL can be used to control access to groups of entries
during search operations, the list ACL can be used to control access
during list operations.  A list ACL may apply to an individual node, or
a node's children.  For example, to prevent everyone except those users
in the US from listing a particular entry, a user might add the
following list ACL to the entry:

\begin{quote}\small\begin{verbatim}
lacl= others # nolist # entry
lacl= prefix # c=US # list # entry
\end{verbatim}\end{quote}

The access selector portion of a list ACL is the same as for a normal
QUIPU ACL.

A list ACL can also be used to control the listing of a node's children.
In addition to specifying whether a particular user can list the
children or not, one can specify the maximum number of children that
will be returned by a single list operation.  For example, to prevent
everyone except US users from listing the children of an entry,
that entry should have the following list ACL:

\begin{quote}\small\begin{verbatim}
lacl= others # nolist # children
lacl= prefix # c=US # list # children
\end{verbatim}\end{quote}

A limit on the number of children returned from a list (10 in this
example) may be imposed by the following:

\begin{quote}\small\begin{verbatim}
lacl= others # list # children # 10
\end{verbatim}\end{quote}

\section{Authentication Policy}
\label{disc_authp}
\index{authPolicy attribute}
\index{authentication policy}

With discretionary access control, search access control, and list access
control, there is a need to authenticate the party requesting access.  It
should be specifiable on a per entry basis what form this authentication
should take for it to be believed.  For example, one trusting individual
might view no authentication as sufficient, allowing access over
unauthenticated DSP links.  Another user might be satisfied with simple
authentication.  Still another security conscious individual might not
be satisfied with anything less than strong authentication.
In addition, there may be different authentication levels required to
perform different operations.  The most common example of this is someone
who will accept no or simple authentication to allow ``read'' access to
their entry, but requires simple or strong authentication to perform
any modifications to their entry.

The authPolicy attribute is used on a per entry basis to provide this
functionality.  It divides access into three categories: modify,
read and compare, and search and list.  For each category, an authentication
policy can be specified.  For example

\begin{quote}\small\begin{verbatim}
authp= strong # simple # trust
\end{verbatim}\end{quote}

requires strong authentication for modification operations, simple
authentication for read and compare operations, and no authentication
for list and search operations on the entry.

The default behavior is as if every entry had the following authPolicy
attribute:

\begin{quote}\small\begin{verbatim}
authp= simple # simple # simple
\end{verbatim}\end{quote}

which requires simple authentication for all operations.

Normally the authPolicy attribute will be inherited throughout an
entire subtree of entries.



\chapter{ISODE Consortium}
\label{IC}

The ISODE Consortium is a not-for-profit cooperative enterprise, whose
mission is to promote and develop the ISODE package of OSI (Open
System Interconnection) applications, which has been used extensively
in the research community.  The ISODE Consortium will be able to
evolve the ISODE software more rapidly than would be possible for any
single member.  This will be to the mutual benefit of members of the
consortium, and will help to stimulate the market for OSI, which
is key technology to enable open communication between and within
organisations.  Membership of the ISODE Consortium is open to any
organisation in any country.

The ISODE Consortium releases of ISODE will be made exclusively
available to the ISODE Consortium members or by purchase of products
from ISODE Consortium members.  Academic organisations, and not for
profit or government organisations with research as their primary
purpose, will be given zero cost access to the ISODE Consortium releases,
on the basis of simply signing a licence with minimal administrative
overhead. This builds on a major strength of the ISODE package by
facilitating use of ISODE within the research community, whilst
allowing ISODE to evolve as a product base.

The ISODE Consortium has mailing lists for bug reports relating to the
ISODE Package.   These are:

\begin{quote}\begin{tabular}{ll}
bug-quipu@isode.com &  -- bugs relating to QUIPU\\
bug-pp@isode.com    &  -- bugs relating to PP\\
bug-isode@isode.com &  -- bugs relating to any other parts of ISODE\\
\end{tabular}\end{quote}

These lists are for the planned ISODE Consortium releases of ISODE.  
Bug reports relating to the ``ISODE-8.0'' release of ISODE are
welcome.   Changes will be incorporated into the ISODE Consortium
releases, and not released to the public domain.  


Further information may be obtained from:-

\begin{tabular}{ll}
ISODE Consortium &               ISODE Consortium\\
US Office, c/o MCC &             European Office\\
P.O. Box 200195    &             P.O. Box 505\\
Austin             &             LONDON\\
TX 78720           &             SW11 1DX\\
USA                &           	 UK\\
Phone: +1-(512)-338-3340       & Phone: +44-71-223-4062\\
Fax:   +1-(512)-338-3600       & Fax:   +44-71-223-3846\\
EMail:  ic-info@isode.com      & EMail:  ic-info@isode.com\\
\end{tabular}

\chapter{Operating System Requirements}
\label{ports}

This appendix contains a table of hardware platforms and
operating systems this version of ISODE is believed to work on.  This
information is based upon reports sent to {\em bug-isode}.  
It is not known how accurate this table is.

\[
\begin{tabular}[h]{||l|c|c||}
\hline
\hline
Machine & OS & Stacks \\
\hline
\hline
\begin{tabular}{l} ??? \end{tabular} & \begin{tabular}{l}  BSD/386 \end{tabular} & \begin{tabular}{l} TCP \end{tabular} \\
\hline
\begin{tabular}{l} CCUR 6000 \end{tabular} & \begin{tabular}{l}  RTU 5.0 \end{tabular} & \begin{tabular}{l} TCP \end{tabular} \\
\hline
\begin{tabular}{l} CCUR 6000 \end{tabular} & \begin{tabular}{l}  RTU 6.0 \end{tabular} & \begin{tabular}{l} TCP\\X25\\CLNS \end{tabular} \\
\hline
\begin{tabular}{l} CDC 4000 Series  \end{tabular} & \begin{tabular}{l} EP/IX 1.3.2\\ EP/IX 1.4.1 \end{tabular} & \begin{tabular}{l} TCP\\CLNS\\X25 \end{tabular} \\
\hline
\begin{tabular}{l} COMPAQ 386/25 \end{tabular} & \begin{tabular}{l} 	SCO Unix 5.2 \end{tabular} & \begin{tabular}{l} TCP \end{tabular} \\
\hline
\begin{tabular}{l} COMPAQ 386 \end{tabular} & \begin{tabular}{l} 	BSD \end{tabular} & \begin{tabular}{l} TCP\\X25 \end{tabular} \\
\hline
\begin{tabular}{l} Convex C120 \end{tabular} & \begin{tabular}{l} 	ConvexOS 8.1 \end{tabular} & \begin{tabular}{l} TCP \end{tabular} \\
\hline
\begin{tabular}{l} DEC Vax \end{tabular} & \begin{tabular}{l} 2nd Berkeley Network rel \end{tabular} & \begin{tabular}{l} TCP\\X25\\CLNS \end{tabular} \\
\hline
\begin{tabular}{l} DEC \end{tabular} & \begin{tabular}{l} 	DECnet-ULTRIX V5.0 \end{tabular} & \begin{tabular}{l} TCP\\CLNS \end{tabular} \\
\hline
\begin{tabular}{l} DEC \end{tabular} & \begin{tabular}{l} 	Ultrix 3.1D\\Ultrix 4.0\\ Ultrix 4.1 \end{tabular} & \begin{tabular}{l} TCP\\X25 \end{tabular} \\
\hline
\begin{tabular}{l} DEC \end{tabular} & \begin{tabular}{l} 	Ultrix 4.2 \end{tabular} & \begin{tabular}{l} TCP\\X25\\CLNS \end{tabular} \\
\hline
\begin{tabular}{l} DEC \end{tabular} & \begin{tabular}{l} 	VMS v5.x \end{tabular} & \begin{tabular}{l} TCP\\X25 \end{tabular} \\
\hline
\begin{tabular}{l} DG Avion \end{tabular} & \begin{tabular}{l}   DGUX 4.30 \end{tabular} & \begin{tabular}{l} TCP \end{tabular} \\
\hline
\begin{tabular}{l} Encore Multimax 3xx\\Encore Multimax 5xx \end{tabular} & \begin{tabular}{l}  UMAX V 2.2h \end{tabular} & \begin{tabular}{l} TCP \end{tabular} \\
\hline
\begin{tabular}{l} Encore NP1 \end{tabular} & \begin{tabular}{l}  UTX/32 3.1a \end{tabular} & \begin{tabular}{l} TCP\\X25 \end{tabular} \\
\hline
\begin{tabular}{l} Encore PN6000\\Encore PN9000 \end{tabular} & \begin{tabular}{l}  UTX/32 2.1b \end{tabular} & \begin{tabular}{l} TCP\\X25 \end{tabular} \\
\hline
\hline
\end{tabular}\\
\]
\[
\begin{tabular}[h]{||l|c|c||}
\hline
\hline
Machine & OS & Stacks \\
\hline
\hline
\begin{tabular}{l} HP/9000/3xx \end{tabular} & \begin{tabular}{l}  HP/UX 6.0\\HP-UX 7.05 B \end{tabular} & \begin{tabular}{l} TCP \end{tabular} \\
\hline
\begin{tabular}{l} HP/9000/8xx \end{tabular} & \begin{tabular}{l}  HP-UX 7.00 \end{tabular} & \begin{tabular}{l} TCP\\X25 \end{tabular} \\
\hline
\begin{tabular}{l} IBM 3090 \end{tabular} & \begin{tabular}{l} 	AIX/370 1.2.1 \end{tabular} & \begin{tabular}{l} TCP \end{tabular} \\
\hline
\begin{tabular}{l} IBM PS/2 \end{tabular} & \begin{tabular}{l} 	AIX 1.2.1 \end{tabular} & \begin{tabular}{l} TCP \end{tabular} \\
\hline
\begin{tabular}{l} IBM RS/6000 \end{tabular} & \begin{tabular}{l} 	AIX 3.2 \end{tabular} & \begin{tabular}{l} TCP\\X25 \end{tabular} \\
\hline
\begin{tabular}{l} ICL \end{tabular} & \begin{tabular}{l}  DRS/6000 \end{tabular} & \begin{tabular}{l} TCP\\X25 \end{tabular} \\
\hline
\begin{tabular}{l} Interactive \end{tabular} & \begin{tabular}{l} Interactive 5.4.0.3 \end{tabular} & \begin{tabular}{l} TCP \end{tabular} \\
\hline
\begin{tabular}{l} Macintosh \end{tabular} & \begin{tabular}{l} A/UX 2.0.1 \end{tabular} & \begin{tabular}{l} TCP \end{tabular} \\
\hline
\begin{tabular}{l} Macintosh \end{tabular} & \begin{tabular}{l} MacOS V6.x \end{tabular} & \begin{tabular}{l} TCP \end{tabular} \\
\hline
\begin{tabular}{l} Mips 4\_52 \end{tabular} & \begin{tabular}{l} 	ATT\_V3\_0 \end{tabular} & \begin{tabular}{l} TCP \end{tabular} \\
\hline
\begin{tabular}{l} NCR 3400 \end{tabular} & \begin{tabular}{l}  SVR4 Unix \end{tabular} & \begin{tabular}{l} TCP \end{tabular} \\
\hline
\begin{tabular}{l} NeXT \end{tabular} & \begin{tabular}{l} 	 \end{tabular} & \begin{tabular}{l} TCP \end{tabular} \\
\hline
\begin{tabular}{l} ORION/Clipper \end{tabular} & \begin{tabular}{l}  \end{tabular} & \begin{tabular}{l} TCP \end{tabular} \\
\hline
\begin{tabular}{l} Olivetti LSX-3020 \end{tabular} & \begin{tabular}{l}  X/OS 2.1 \end{tabular} & \begin{tabular}{l} TCP\\X25 \end{tabular} \\
\hline
\begin{tabular}{l} Pyramid 9800\\ Pyramid MIS \end{tabular} & \begin{tabular}{l}  OSx 5.1 (4.3BSD/SVR3.2) \end{tabular} & \begin{tabular}{l} TCP \end{tabular} \\
\hline
\begin{tabular}{l} SEQUENT \end{tabular} & \begin{tabular}{l}   DYNIX V3.0.18 \end{tabular} & \begin{tabular}{l} TCP \end{tabular} \\
\hline
\begin{tabular}{l} Silicon Graphics IRIS \end{tabular} & \begin{tabular}{l}  IRIS 3.2.2 \end{tabular} & \begin{tabular}{l} TCP \end{tabular} \\
\hline
\begin{tabular}{l} Silicon Graphics IRIS \end{tabular} & \begin{tabular}{l}  IRIS 4.01 \end{tabular} & \begin{tabular}{l} TCP \end{tabular} \\
\hline
\begin{tabular}{l} Solbourne Series 5/600 \end{tabular} & \begin{tabular}{l}  OS/MP 4.1 \end{tabular} & \begin{tabular}{l} TCP \end{tabular} \\
\hline
\begin{tabular}{l} Sony News-1750 \end{tabular} & \begin{tabular}{l} NEWS-OS 3.3\\NEWS-OS 4.0c \end{tabular} & \begin{tabular}{l} TCP \end{tabular} \\
\hline
\begin{tabular}{l} Sony News-3250 \end{tabular} & \begin{tabular}{l} System V.4 \end{tabular} & \begin{tabular}{l} TCP \end{tabular} \\
\hline
\begin{tabular}{l} Sun4\\Sun3 \end{tabular} & \begin{tabular}{l} SunOS 4.1\\SunOS 4.1.1\\SunOS 4.1.2\\SunOS 4.0.3c  \end{tabular} & \begin{tabular}{l} TCP\\X25\\CONS\\CLNS \end{tabular} \\
\hline
\hline
\end{tabular}\\[3ex]
\]




\end{document}

